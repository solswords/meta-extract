% General comments and "To Dos":

% I've fixed right margins -- well, the worst offender is line 3 of
% Section 2.1, but I think that's minor and it seems awkward to fix.
% I've also done a full spell-check.  If we make further changes we
% could do those again.

% Reviewer 2 says the following.
%%% Presumably the ACL2 source code essay on correctness addresses the
%%% soundness concerns related to meta-extract.  Even so, I think this
%%% paper should at least address the most basic question: why not
%%% simply axiomatize the appropriate functional interfaces to ACL2?
%%% Presumably because that is not sound, but the paper should at
%%% least explain why and then contrast that with the preferred
%%% approach.
% I'm not completely sure that I understand, but I think the point is
% that instead of something like this: instead of using
% meta-extract-contextual-fact in our first example, why not have a
% hypothesis something like this (which is equivalent, as we show)?
%   (typespec-check
%    (mfc-ts (cadr term) mfc state :forcep nil)
%    (nthmeta-ev (cadr term) a))
% Do you think that's what the reviewer is asking?  If so, do you have
% an answer?  I'm being lazy here; maybe I could think harder about
% this, but I suspect you'll know right away.  If not let me know and
% I'll think some more.  Similarly I'm hoping to punt to you on the
% other two questions from that reviewer; would you take a look?  Ah,
% except I might have some answer to the last part starting with "Sol
% has apparently...": this stuff is tricky to get right, and it seemed
% more straightforward to implement what we did and then let a book
% take care of the universal quantification.  Should I write that up,
% or isn't that quite the right answer?

% Reviewer 3 says that following, but I think we should ignore it.
%%% Section 1.1 Perhaps the referenced essay in the source code should
%%% be published -- it seems important.

% I checked references for which DOIs are missing and I didn't find
% any more.  As far as I'm concerned we're good on the references, and
% EPTCS can complain if it wants to.

% End of General comments and "To Dos".

\documentclass[submission,copyright,creativecommons]{eptcs}
\providecommand{\event}{ACL2 Workshop 2017} % Name of the event you are submitting to
% \usepackage{breakurl}             % Not needed if you use pdflatex only.
\usepackage{amsmath,amssymb,amsfonts,amsthm}
\usepackage{fancyvrb}
% so we can use \-/ for breakable dashes in long names
\usepackage[shortcuts]{extdash}
% \usepackage{bigfoot}
% \usepackage{tikz}
\usepackage{color}
\usepackage{framed}

% \newtheorem{theorem}{Theorem}
% \newtheorem{corollary}{Corollary}

% The first of the following two definitions of the comment
% environment is adapted from
% http://tex.stackexchange.com/questions/71371/how-to-change-color-in-an-environment
% while the second comes from a package.  Choose the first to see
% comments and the second to hide them.

\newenvironment{comment}
  {\par\medskip
   \color{red}%
   \begin{framed}
   \ignorespaces}
 {\end{framed}
  \medskip}

% \usepackage{comment}

% Neither of the following worked for me:

%\newenvironment{mycomment}{{\begin{comment}}}{{\end{comment}}}

% \let\mycomment\comment
% \let\endmycomment\endcomment

\usepackage{listings}
\lstset{ %
  basicstyle=\normalsize\ttfamily,
  language=lisp,
  columns=fullflexible,
  escapeinside={\<}{\>},
}

\newcommand*{\var}[1]{\mathit{#1}}

% The following comment is from Reviewer 3:
%%% The title is short-and-sweet, but perhaps too short.  It would be
%%% nice if the title conveyed more information to the ACL2-curious
%%% user of other theorem provers; something like "Using Meta-extract
%%% to Craft Custom Proof Routines in ACL2" (using language from the
%%% paper).
% I don't like the suggested new title, because it might suggest that
% this paper is about the addition of meta rules.  I was tempted to
% make the title "Meta-extract: Reflecting What We Know", but that's
% too obscure.  Maybe: "Meta-extract: Bringing Implicit Knowledge to
% Bear".  I dunno.  Ideas?

\title{Meta-extract}
\author{Matt Kaufmann
\institute{Department of Computer Science\\
The University of Texas at Austin\\
Austin, TX, USA}
\email{kaufmann@cs.utexas.edu}
\and
Sol Swords
\institute{Centaur Techology, Inc.\\
Austin, TX, USA}
\email{sswords@centtech.com}
}
\def\titlerunning{Meta-extract}
\def\authorrunning{M. Kaufmann \& S. Swords}
\begin{document}
\maketitle

\begin{abstract}

  ACL2 has long supported user-defined simplifiers, so-called {\em
    metafunctions} and {\em clause processors}, which are installed
  when corresponding rules of class {\tt :meta} or {\tt
    :clause-processor} are proved.  Historically, such
  simplifiers could access the logical world at execution time and
  could call certain built-in proof tools, but could not assume the
  soundness of the proof tools or the truth of any facts extracted
  from the world or context when proving a simplifier correct.  Starting with
  ACL2 Version 6.0, released in December 2012, an additional
  capability was added which allows the correctness proofs of
  simplifiers to assume the correctness of some such proof tools and
  extracted facts.  In this paper we explain this capability and give
  examples that demonstrate its utility.

\end{abstract}

\section{Introduction}
\label{sec:intro}


ACL2's meta rule and clause processor facilities are designed to allow
users of ACL2 to write custom proof routines which, once proven
correct, can be called by the ACL2 prover\footnote{Alternatively, such
  proof routines may be ``trusted''---used without proof, but at the
  cost of a ``trust tag'' marking this use as a source of
  unsoundness.}.

Such proof routines have historically been limited in their use of
ACL2's database of stored facts and its built-in prover functions.
For example, a metafunction could call the ACL2 rewriter, but when
proving this metafunction correct, the result of calling the rewriter
could not be assumed to be equivalent to the input --- that is, the
rewriter could not be assumed to be correct.  Somewhat similarly,
metafunctions and clause processors could both examine the ACL2 world
(the logical state of the prover, including the database of stored
facts), but in proving the correctness of these functions, facts
extracted from the world could not be assumed to be correct.

Meta-extract is an ACL2 feature first introduced in Version 6.0
(December, 2012).  It allows for certain facts stored in the ACL2
world and certain ACL2 prover routines to be effectively assumed
correct when proving the correctness of metafunctions and clause
processors.

Correctness proofs for metafunctions and clause processors reference a
function that we will call a \textit{pseudo-evaluator}, typically
defined via the \texttt{defevaluator} macro.  (Elsewhere in the ACL2
literature this is simply referred to as an evaluator; however, we
want to emphasize the difference between a pseudo-evaluator and a true
term evaluator.)  A pseudo-evaluator is a constrained function about
which only certain facts are known; these facts are ones that would
also be true of a ``real'' evaluator capable of fully interpreting any
ACL2 term.  It is hoped that these facts are sufficient to allow the
user to prove a metafunction or clause-processor correct as well as if
they could use a real term evaluator.  To prove a metafunction correct
the user must show that the pseudo-evaluation of the term output by
the metafunction is equal (or equivalent) to the pseudo-evaluation of
the input term.  Intuitively, if this can be proved of a
pseudo-evaluator, and the only facts known about the pseudo-evaluator
are ones that are also true of the real evaluator of ACL2 terms, then
this must also be true of the real evaluator: that is, the evaluations
of the input and output terms of the metafunction are equivalent, and
thus the metafunction is correct.

The meta-extract feature allows certain hypotheses to be assumed
during the proof of correctness of a metafunction.  These meta-extract
hypotheses are applications of the pseudo-evaluator to calls of the
functions
\texttt{meta-extract-contextual-fact} and
\texttt{meta-extract-global-fact+}.  These functions produce various
sorts of terms by extracting facts from the ACL2 world and calling
ACL2 prover subroutines, constructed so that if ACL2 and its logical
state is sound, the terms produced should always be true.  For some
examples, these functions can produce:
\begin{itemize}
\item the body of a previously proven theorem
\item the definitional equation of a previously defined function
\item a term equating a call of a function on quoted constants to the
  quoted value of that call
\item a term equating some term $a$ to the result of rewriting $a$
\item a term describing the typeset of $a$, according to ACL2's type
  reasoning.
\end{itemize}

Intuitively, adding a meta-extract hypothesis to a metafunction's
correctness theorem is allowable because we expect the (real)
evaluation of any term produced by one of the meta-extract functions
to return true.  If we prove the pseudo-evaluator theorem with a
meta-extract hypothesis and, as before, reason that since the theorem
is true of the pseudo-evaluator, it is also true of the real
evaluator, then the final step is to say that the meta-extract
hypothesis using the real evaluator is true (or else ACL2 is already
unsound).  Therefore we can still conclude that the evaluations of the
metafunction's output and input terms are equivalent.




\section{Meta-extract}
\label{sec:meta-extract}
Next we explain meta-extract by first giving two examples and then
summarizing the general forms of meta-extract hypotheses.

\subsection{Tutorial Examples}

We present two examples for the two kinds of meta-extract hypotheses,
corresponding to evaluation of calls of {\tt
  meta-extract-contextual-fact} and of {\tt meta-extract-global-fact}.
(Since a call {\tt (meta-extract-global-fact obj state)} abbreviates
the call {\tt (meta-extract-global-fact+ obj state state)}, we are
thus effectively illustrating {\tt meta-extract-global-fact+} as
well.)

\subsubsection{Meta-extract-contextual-fact}

Our first example is intentionally contrived and quite trivial,
intended only to provide an easy introduction to meta-extract.  It
illustrates the use of {\tt meta-extract-contextual-fact}.  The intent
is to simplify any term of the form {\tt (nth $x$ $lst$)}, when $x$ is
easily seen by ACL2 to be a symbol in the current context, to {\tt
  (car $lst$)}.

In ACL2, the use of metafunctions is always supported by an evaluator,
called a {\em pseudo-evaluator} in the preceding section.  Let us
introduce an evaluator that ``knows'' about the functions relevant to
this example.

\begin{verbatim}
(defevaluator nthmeta-ev nthmeta-ev-lst
  ((typespec-check ts x)
   (nth n x)
   (car x)))
\end{verbatim}

\noindent Next we define a metafunction, intended to replace any term
{\tt (nth n x)} by a corresponding term {\tt (car x)} when {\tt n} is
known to be a symbol using
\href{http://www.cs.utexas.edu/users/moore/acl2/manuals/current/manual/index.html?topic=ACL2\_\_\_\_TYPE-SET}{\underline{type-set}}
reasoning.

\begin{verbatim}
(defun nth-symbolp-metafn (term mfc state)
  (declare (xargs :stobjs state))
  (case-match term
    (('nth n x)
     (if (equal (mfc-ts n mfc state :forcep nil)
                *ts-symbol*)
         (list 'car x)
       term))
    (& term)))
\end{verbatim}

\noindent When the input term matches {\tt (nth $n$ $x$)}, this calls
\href{http://www.cs.utexas.edu/users/moore/acl2/manuals/current/manual/index.html?topic=ACL2\_\_\_\_MFC-TS}{\underline{\tt
    mfc-ts}}
to deduce the possible types of $n$.  If that type-set equals {\tt
  *ts-symbol*} then that term must evaluate to a symbol, and the term
can be reduced to {\tt (car $x$)}.\footnote{This is an unnecessarily strong check: it means ACL2's determination
was that the possible types of $n$ included all three of the basic
types {\tt T}, {\tt NIL}, and non-Boolean symbols.  If ACL2 was able
to rule out any of those three possibilities, the function would
fail to simplify the term.}

Now we can present a meta rule with a meta-extract
hypothesis.  Without that hypothesis the formula below is
not a theorem, because the function {\tt mfc-ts} has no axiomatic
properties; all we know about it below is what we are told by the
meta-extract hypothesis, as discussed further below.

\begin{verbatim}
(defthm nth-symbolp-meta
    (implies (nthmeta-ev (meta-extract-contextual-fact `(:typeset ,(cadr term))
                                                       mfc
                                                       state)
                         a)
             (equal (nthmeta-ev term a)
                    (nthmeta-ev (nth-symbolp-metafn term mfc state) a)))
    :rule-classes ((:meta :trigger-fns (nth))))
\end{verbatim}

\noindent To see what the meta-extract hypothesis above gives us, consider the
following theorem provable by ACL2.

\begin{verbatim}
(equal (meta-extract-contextual-fact `(:typeset ,x)
                                      mfc
                                      state)
       (list 'typespec-check
             (list 'quote
                   (mfc-ts x mfc state :forcep nil))
             x))
\end{verbatim}

\noindent At a high level, this theorem shows us that {\tt
  meta-extract-contextual-fact} returns a term {\tt
  (\href{http://www.cs.utexas.edu/users/moore/acl2/manuals/current/manual/index.html?topic=ACL2\_\_\_\_TYPESPEC-CHECK}{\underline{typespec-check}} (quote $ts$) x)}, which asserts that the term {\tt
  x} belongs to the set of values represented by $ts$.  The
meta-extract hypothesis applies the pseudo-evaluator to this term, and
since {\tt typespec-\allowbreak{}check} is one of its known functions, the
hypothesis reduces to
\begin{verbatim}
  (typespec-check (mfc-ts (cadr term) mfc state :forcep nil)
                  (nthmeta-ev (cadr term) a)).
\end{verbatim}
\noindent The interesting case in proving our metafunction correct is
when this {\tt mfc-ts} call equals {\tt *ts-symbol*}.  In this case the hypothesis becomes
\begin{verbatim}
  (typespec-check *ts-symbol* (nthmeta-ev (cadr term) a))
\end{verbatim}
\noindent which, when expanded, implies that the evaluation of {\tt
  (cadr term)} must be a
symbol, which enables the proof of {\tt nth-symbolp-meta}.

% \begin{comment}
%   Do we need to discuss the application of the meta rule?  As
%   indicated in the introduction, nothing about the application of
%   metafunctions or clause processors has changed.

%   [Matt] Excellent point.  I've made this much shorter and explained
%   that it's just business as usual, since I think that's worth
%   emphasizing.  But if you want to delete all of it, please feel free;
%   in particular, we should delete it without a moment's thought if we
%   need the space.
% \end{comment}

Recall that meta-extract hypotheses do not affect the {\em
  applications} of meta rules; they only support their proofs.
Therefore, the following test of the example above yields no surprises.

\begin{verbatim}
(defstub foo (x) t)
(thm (implies (symbolp (foo x))
              (equal (nth (foo x) y) (car y)))
     :hints (("Goal" :in-theory '(nth-symbolp-meta))))
\end{verbatim}

%%% \noindent Indeed, the summary shows the use of the meta rule, {\tt
%%%   nth-symbolp-meta}.  We can even trace our metafunction: after
%%% evaluating the form
%%%
%%% \begin{verbatim}
%%% (trace$ (nth-symbolp-metafn
%%%          :entry
%%%          (list 'nth-symbolp-metafn term '<mfc> 'state)))
%%% \end{verbatim}
%%%
%%% \noindent we see the following in the proof log for the {\tt thm} call
%%% above.
%%%
%%% \begin{verbatim}
%%% 1> (NTH-SYMBOLP-METAFN (NTH (FOO X) Y)
%%%                        <MFC> STATE)
%%% <1 (NTH-SYMBOLP-METAFN (CAR Y))
%%% \end{verbatim}

\subsubsection{Meta-extract-global-fact}

Our second example is from community book
  ``demos/nth-update-nth-meta-extract.lisp'', which uses {\tt
  meta-extract-global-fact}.  Let us begin by seeing what problem this
book is attempting to solve.

% \begin{comment}
%   I like this section but it's pretty long --- about 3.5 pages.  Could
%   we omit the details about the macro that generalizes this to an
%   arbitrary stobj and just assert that it's easy to do?  I think most
%   people who will understand this paper could figure out how to do
%   that.

%   [Matt] Good idea -- I've done that, leaving only a brief description
%   of the macro.
% \end{comment}

Consider a
\href{http://www.cs.utexas.edu/users/moore/acl2/manuals/current/manual/index.html?topic=ACL2\_\_\_\_DEFSTOBJ}{\underline{\tt
    defstobj}} event, {\tt (defstobj st fld$_1$ fld$_2$ ... fld$_n$)}.
A common challenge in reasoning about stobjs is the simplification of
{\em read-over-write} terms, of the form {\tt (fld$_i$ (update-fld$_j$
  $v$ st))}, which indicate that we are to read {\tt fld$_i$} after
updating {\tt fld$_j$}.  That term simplifies to {\tt (fld$_i$ st)}
when $i \neq j$, and otherwise it simplifies to $v$.  How do we get
ACL2 to do such simplification automatically?  The following two
approaches are standard.

\begin{itemize}

\item Disable the stobj accessors and updaters after proving
  rewrite rules to simplify terms of the form {\tt (fld$_i$
    (update-fld$_j$ val st))}, to {\tt val} if $i = j$ and to {\tt
    (fld$_i$ st)} if $i \neq j$.

\item Let the stobj accessors and updaters remain enabled, relying on
  a rule such as the built-in rewrite rule {\tt nth-update-nth} to
  rewrite terms, obtained after expanding calls of the accessors and
  updaters, of the form {\tt (nth $i$ (update-nth $j$ val st))}.

\end{itemize}

\noindent The first of these requires $n^2$ rules, which is generally feasible
but can perhaps get somewhat unwieldy.  The second of these provides a
simple solution, but when proofs fail, the resulting checkpoints can
be more difficult to comprehend.

Here, we outline a solution that addresses both of these concerns: a
macro that generates a suitable meta rule.  Details of this proof
development may be found in community book
``demos/\allowbreak{}nth-update-nth-meta-extract.lisp''.
First we introduce our metafunction.  Next, we prove a meta rule for a
specific stobj.  Finally, we mention a macro that generates a version
of this rule that is suitable for an arbitrary specified stobj.

Our metafunction returns the input term unchanged unless it is of the
form {\tt ($r$ ($w$ $v$ $x$))}, where $r$ and $w$ are {\em reader}
(accessor) and {\em writer} (updater) functions defined to be calls of
{\tt nth} and {\tt update-nth}, on explicit indices: {\tt ($r$ $x$)}
$=$ {\tt (nth '$i$ $x$)} and {\tt ($w$ $v$ $x$)} $=$ {\tt (update-nth
  '$i$ $v$ $x$)}.  In that case, the function {\tt
  nth-update-nth-meta-fn-new-term} computes a new term: $v$ if $i =
j$, and otherwise, {\tt ($r$ $x$)}.

\begin{verbatim}
(defun nth-update-nth-meta-fn (term mfc state)
  (declare (xargs :stobjs state)
           (ignore mfc))
  (or (nth-update-nth-meta-fn-new-term term state)
      term))
\end{verbatim}

Notice below that in computing the new term, the definitions of the
reader and writer are extracted from the logical world using the
function, {\tt meta-extract-formula}, which returns the function's
definitional equation.  For example:

\begin{verbatim}
ACL2 !>(meta-extract-formula 'atom state)
(EQUAL (ATOM X) (NOT (CONSP X)))
ACL2 !>
\end{verbatim}

\noindent We thus rely on the correctness of {\tt
  meta-extract-formula} for the equality of the input term and the
term returned by the following function.

\begin{Verbatim}[commandchars=\\\{\},fontsize=\small]
(defun nth-update-nth-meta-fn-new-term (term state)
  (declare (xargs :stobjs state))
  (case-match term
    ((reader (writer val x))
     (and (not (eq reader 'quote))
          (not (eq writer 'quote))
          (let* ((reader-formula (and (symbolp reader)
                                      (meta-extract-formula reader state)))
                 (i-rd (fn-nth-index reader reader-formula)))
            (and
             i-rd {\em \color{red} ; the body of reader is (nth 'i-rd ...)}
             (let* ((writer-formula (and (symbolp writer)
                                         (meta-extract-formula writer state)))
                    (i-wr (fn-update-nth-index writer writer-formula)))
               (and
                i-wr {\em \color{red} ; the body of writer is (update-nth 'i-wr ...)}
                (if (eql i-rd i-wr)
                    val
                  (list reader x))))))))
    (& nil)))
\end{Verbatim}

Next we introduce a (pseudo-)evaluator to use in our meta rule.

\begin{verbatim}
(defevaluator nth-update-nth-ev nth-update-nth-ev-lst
  ((nth n x)
   (update-nth n val x)
   (equal x y)))
\end{verbatim}

We need to define one more function before presenting our meta rule.
It takes as input a term {\tt ($f$ $t_1$ $\ldots$ $t_n$)} with list of
formals {\tt ($v_1$ $\ldots$ $v_n$)}, and builds an alist that maps
each $v_i$ to the value of $t_i$ in a given alist.  Like our
metafunction, it consults {\tt meta-extract-formula} to obtain the
formal parameters of $f$.  An earlier attempt to use the function,
{\tt formals}, failed: the meta rule's proof needs these formals to
connect to those found by our metafunction.

\begin{Verbatim}[commandchars=\\\{\},fontsize=\small]
(defun meta-extract-alist-rec (formals actuals a)
  (cond ((endp formals) nil)
        (t (acons (car formals)
                  (nth-update-nth-ev (car actuals) a)
                  (meta-extract-alist-rec (cdr formals) (cdr actuals) a)))))

(defun meta-extract-alist (term a state)
  (declare (xargs :stobjs state :verify-guards nil))
  (let* ((fn (car term))
         (actuals (cdr term))
         (formula (meta-extract-formula fn state)) {\em \color{red} ; (equal (fn ...) ...)}
         (formals (cdr (cadr formula))))
    (meta-extract-alist-rec formals actuals a)))
\end{Verbatim}

\noindent We now define a stobj and prove a corresponding meta rule.

\begin{Verbatim}[commandchars=\\\{\},fontsize=\small]
(defstobj st
  fld1 fld2 fld3 fld4 fld5 fld6 fld7 fld8 fld9 fld10
  fld11 fld12 fld13 fld14 fld15 fld16 fld17 fld18 fld19 fld20)

(defthm nth-update-nth-meta-rule-st
  (implies
   (and (nth-update-nth-ev {\em \color{red} ; (f (update-g val st))}
         (meta-extract-global-fact (list :formula (car term)) state)
         (meta-extract-alist term a state))
        (nth-update-nth-ev {\em \color{red} ; (update-g val st)}
         (meta-extract-global-fact (list :formula (car (cadr term)))
                                   state)
         (meta-extract-alist (cadr term) a state))
        (nth-update-nth-ev {\em \color{red} ; (f st) -- note st is (caddr (cadr term))}
         (meta-extract-global-fact (list :formula (car term)) state)
         (meta-extract-alist (list (car term)
                                   (caddr (cadr term)))
                             a state)))
   (equal (nth-update-nth-ev term a)
          (nth-update-nth-ev (nth-update-nth-meta-fn term mfc state) a)))
  :hints ...
  :rule-classes ((:meta :trigger-fns (fld1 fld2 ... fld20))))
\end{Verbatim}

\noindent The proof of the theorem below takes no measurable time, and
applies the metafunction proved correct above.

\begin{verbatim}
   (in-theory (disable fld1 ... fld20 update-fld1 ... update-fld20))

   (defthm test1
     (equal (fld3 (update-fld1 1
                   (update-fld2 2
                    (update-fld3 3
                     (update-fld4 4
                      (update-fld3 5
                       (update-fld6 6 st)))))))
            3))
\end{verbatim}

Notice that there is nothing about the meta rule above that is
specific to the particular stobj, {\tt st}, except for the {\tt
  :trigger-fns} that it specifies.  In the community book mentioned
above (``demos/nth-\allowbreak{}update-\allowbreak{}nth-\allowbreak{}meta-\allowbreak{}extract.lisp''), we define a macro
that automates the generation of such a meta rule for an
arbitrary stobj.  Our macro takes the name of a stobj, $s$, and does two
things: it disables all of the stobj's accessors and updaters, and it
proves a meta rule that simplifies every term of the form {\tt ($r$
  ($w$ $v$ $s$))}, where $r$ and $w$ are an accessor and updater,
respectively, for the stobj $s$.
%%% We omit some details here, including
%%% the definition of function {\tt stobj-accessors-and-updaters}, which
%%% as its name suggests, returns a list of all stobj accessors and
%%% updaters for a given stobj.
%%%
%%% \begin{Verbatim}[commandchars=\\\{\},fontsize=\small]
%%% (defthm nth-update-nth-meta-level
%%%   {\em{<same formula as in meta rule above>}}
%%%   :hints (("Goal" :in-theory (enable nth-update-nth-ev-constraint-0)))
%%%   :rule-classes nil)
%%%
%%% (defmacro make-nth-update-nth-meta-stobj (stobj-name)
%%%   `(make-event
%%%     (let ((fns (and (symbolp ',stobj-name)
%%%                     (stobj-accessors-and-updaters ',stobj-name (w state))))
%%%           (theorem-name ...))
%%%       (cond
%%%        ((null fns) {\em <error>})
%%%        (t (value `(progn
%%%                     (in-theory (disable ,@fns))
%%%                     (defthm ,theorem-name
%%%                       {\em{<same formula as in meta rule above>}}
%%%                       :hints (("Goal" :by nth-update-nth-meta-level))
%%%                       :rule-classes ((:meta :trigger-fns ,fns))))))))))
%%% \end{Verbatim}
%%%
%%% The following test succeeds, proving almost instantly by application
%%% of our metafunction.
%%%
%%% \begin{verbatim}
%%% (defstobj st$
%%%   fld$1 fld$2 fld$3 fld$4 fld$5 fld$6 fld$7 fld$8 fld$9 fld$10)
%%%
%%% (make-nth-update-nth-meta-stobj st$)
%%%
%%% (defthm test2
%%%   (equal (fld$3 (update-fld$1 1
%%%                  (update-fld$2 2
%%%                   (update-fld$3 3
%%%                    (update-fld$4 4
%%%                     (update-fld$3 5
%%%                      (update-fld$6 6 st$)))))))
%%%          3))
%%% \end{verbatim}

\subsection{General Forms}
\label{sec:general}

In this section we summarize briefly the forms of meta-extract
hypotheses.  Additional details may be found in the documentation for
\href{http://www.cs.utexas.edu/users/moore/acl2/manuals/current/manual/index.html?topic=ACL2\_\_\_\_META-EXTRACT}{\underline{META-EXTRACT}}.
% These forms correspond to the theorems described at the end of
% Section~\ref{sec:user}.

Below, let {\tt evl} be the pseudo-evaluator (see
Section~\ref{sec:intuitive}) used in a meta rule.  The two primary
forms of meta-extract hypotheses that can be used in that theorem are
as follows.  In the first, {\tt aa} represents an arbitrary term; in
the second, {\tt a} must be the second argument of the two calls of
the (pseudo-)evaluator in the conclusion of the theorem.

\begin{verbatim}
    (evl (meta-extract-global-fact obj state) aa)
    (evl (meta-extract-contextual-fact obj mfc state) a)
\end{verbatim}

The second form above is only legal for meta rules about extended
metafunctions (which take arguments {\tt mfc} and {\tt state}).  The
first form above is actually equivalent to the first form below, which
in turn is a special case of the second form below.

\begin{verbatim}
    (evl (meta-extract-global-fact+ obj state state) aa)
    (evl (meta-extract-global-fact+ obj st state) aa)
\end{verbatim}

The last form supports clause processors that modify state as long as
they do not change the logical world; it produces the same value as
the previous form as long as both {\tt st} and {\tt state} have equal
world fields.

If the arguments to the meta-extract-* function are somehow malformed,
then it returns the trivial term {\tt 'T}, which is not of any use in
proving a metafunction correct.  Otherwise, each invocation produces a
term that states the ``correctness'' of an invocation of some
utility, such as {\tt mfc-rw+} or {\tt meta-\allowbreak{}extract-\allowbreak{}formula}.  The
terms produced for different kinds of invocation vary according to the
particular concept of correctness appropriate to the utility in
question.


%%% \begin{comment}
%%%   I think it would be nice to explain a little bit more in these
%%%   listings rather than just describing exactly the term that each
%%%   invocation produces.  Ideally I think we should convey for each form:
%%%   \begin{itemize}
%%%   \item what utility it allows metafunctions to use
%%%   \item how the term that is produced implies the ``correctness'' of
%%%     an invocation of that utility
%%%   \item why we believe the term produced to always be true.
%%%   \end{itemize}
%%%   I've done a pass to move toward this, but you can revert to your
%%%   previous version if you'd rather continue from that point...
%%% \end{comment}
% -- Agreed!  Nicely done; thanks.

We now describe the allowed {\tt obj} arguments to {\tt
  meta-extract-global-fact} and the terms they produce.  The
documentation for
\href{http://www.cs.utexas.edu/users/moore/acl2/manuals/current/manual/index.html?topic=ACL2\_\_\_\_EXTENDED-METAFUNCTIONS}{\underline{extended-metafunctions}}
explains meta-level functions discussed below, such as {\tt mfc-rw}
and {\tt mfc-ap}.

\begin{itemize}

\item {\tt (:formula $f$)} produces the value of {\tt
    (meta-extract-formula $f$ state)}, which allows metafunctions to
  assume that any invocation of {\tt meta-extract-formula} produces a
  true formula.  {\tt Meta-extract-formula} looks up various kinds of
  formulas from the world:
  \begin{itemize}
  \item the body of $f$ if it is a theorem name
  \item the definitional equation of $f$ if it is a defined function
  \item the constraint of $f$ if it is a constrained function
  \item the defchoose axiom of $f$ if it is a \href{http://www.cs.utexas.edu/users/moore/acl2/manuals/current/manual/index.html?topic=ACL2\_\_\_\_DEFCHOOSE}{\underline{\tt defchoose}} function.
  \end{itemize}

\item {\tt (:lemma $f$ $n$)} produces a term corresponding to the
  $n$th rewrite rule stored in the {\tt lemmas} property of function
  $f$, which allows any such rule to be assumed correct in a
  metafunction.  The term returned is the value of
  \begin{lstlisting}[basicstyle=\linespread{0.4}\normalsize\ttfamily]
    (rewrite-rule-term (nth '<$n$> (getpropc '<$f$> 'lemmas nil (w state))))<\textrm{,}>
  \end{lstlisting}
  assuming that $n$ is a valid index (does not exceed
  the length of the indicated list).  Here {\tt rewrite-\allowbreak{}rule-\allowbreak{}term}
  transforms a {\tt rewrite-rule} record structure into a term such as
  \begin{lstlisting}[basicstyle=\linespread{0.4}\normalsize\ttfamily]
    (implies <$\var{hyps}$> (<$\var{equiv}$> <$\var{lhs}$> <$\var{rhs}$>))<\textrm{.}>
  \end{lstlisting}

\item {\tt (:fncall $f$ $L$)} produces a term {\tt (equal $c$ '$v$)}, where
    $c$ is the call that applies $f$ to the quotations of the values in argument list $L$, and $v$ is
    the value of that call computed by
    \href{http://www.cs.utexas.edu/users/moore/acl2/manuals/current/manual/index.html?topic=ACL2\_\_\_\_MAGIC-EV-FNCALL}{\underline{\tt
        magic-\allowbreak{}ev-\allowbreak{}fncall}}.  This allows
    metafunctions to assume that {\tt magic-ev-fncall}
    correctly evaluates function applications.

\end{itemize}

The allowed {\tt obj} arguments to {\tt meta-extract-contextual-fact} are as follows.

\begin{itemize}

\item {\tt (:typeset $\var{term}$)} produces a term stating the correctness of
    the type-set produced by {\tt mfc-ts} for $\var{term}$.  Specifically, it produces the term
    {\tt (typespec-check '$\var{ts}$ $\var{term}$)}, where $\var{ts}$ is
    the result of {\tt (mfc-ts $\var{term}$ mfc state :forcep nil :ttreep nil)} and
    {\tt (typespec-check ts val)} is true when {\tt val}'s actual type is in the type-set {\tt ts}.

  \item {\tt (:rw+ $\var{term}\ \var{alist}\ \var{obj}\ \var{equiv}$)}
    produces a term stating that {\tt mfc-rw+} correctly rewrites
    $\var{term}$ under substitution $\var{alist}$ with objective
    $\var{obj}$ under equivalence relation $\var{equiv}$.  The form of the term produced is
  \begin{lstlisting}[basicstyle=\linespread{0.4}\normalsize\ttfamily]
    (<$\var{equiv}$> <$\var{term}'$> <$\var{rw}$>)
  \end{lstlisting}
  where $\var{term}'$ is the new term formed by substituting
  $\var{alist}$ into $\var{term}$ and $\var{rw}$ is the result of the call
  {\tt (mfc-rw+ $\var{term}\ \var{alist}\ \var{obj}\ \var{equiv}$ mfc state :forcep nil :ttreep nil)}.
  (Actually, the $\var{equiv}$ argument may also be {\tt T} meaning
  {\tt IFF} or {\tt NIL} meaning {\tt EQUAL}.)

\item {\tt (:rw $\var{term}\ \var{obj}\ \var{equiv}$)} is similar to the
  {\tt :rw+} form above, but instead of {\tt mfc-rw+} it supports {\tt
    mfc-rw}, which takes no $\var{alist}$ argument.  Instead, {\tt
    NIL} is used for the substitution.

\item {\tt (:ap $\var{term}$)} uses {\tt mfc-ap} to derive a linear
  arithmetic contradiction indicating that $\var{term}$ is false, and
  produces {\tt (not $\var{term}$)} if that is successful, that is, if
  {\tt (mfc-ap $\var{term}$ mfc state :forcep nil)} returns true;
  otherwise it just produces {\tt 'T}.

\item {\tt (:relieve-hyp $\var{hyp}\ \var{alist}\ \var{rune}\
    \var{target}\ \var{backptr}$)} uses {\tt mfc-relieve-hyp} to
  attempt to prove that $\var{hyp}$ holds under substitution
  $\var{alist}$, and produces the substitution of
  $\var{alist}$ into $\var{hyp}$ if successful, that is, if
  {\tt (mfc-relieve-hyp $\var{hyp}\ \var{alist}\ \var{rune}\
    \var{target}\ \var{backptr}$ mfc state :forcep nil :ttreep nil)}
  returns true; otherwise, {\tt 'T}.

\end{itemize}

%%%\begin{comment}
%%%
%%%  We had discussed including some sort of correctness argument here.
%%%  But Section~\ref{sec:intuitive} already gives a nice high-level
%%%  correctness argument, and a detailed argument is given in the ACL2
%%%  sources, in the Essay on Correctness of Meta Reasoning.  With some
%%%  effort I could probably write a few remarks here on some of the
%%%  issues addressed in that correctness argument.  But I'm not sure
%%%  anyone would care.  Instead,
%%%
%%%  [Sol] -- I agree; I don't think we need to include a more detailed
%%%  proof of correctness here.  If we do our job right above then
%%%  readers can see why we believe each term produced by the
%%%  meta-extract-* functions to be true.
%%%\end{comment}


\section{Using ``meta-extract-user.lisp''}
\label{sec:user}
The community book ``clause-processors/meta-extract-user.lisp'' is
designed to allow more convenient use of the meta-extract facility.
The main contribution of this book is in the event-generating macro
\texttt{def\-/meta\-/extract}. For a given pseduo-evaluator \texttt{evl},
\texttt{def\-/meta\-/extract} produces macros
\texttt{evl\-/meta\-/extract\-/contextual\-/facts} and
\texttt{evl\-/meta\-/extract\-/global\-/facts} that expand to meta-extract
hypotheses where the \texttt{obj} argument is a call of a ``bad-guy''
function.  This essentially universally quantifies the \texttt{obj}
argument:
\begin{verbatim}
  (evl (meta-extract-contextual-fact
         (evl-meta-extract-bad-guy a mfc state)
         mfc state)
       a)
\end{verbatim}
implies for any \texttt{obj},
\begin{verbatim}
  (evl (meta-extract-contextual-fact obj state) a).
\end{verbatim}

The \texttt{def-meta-extract} utility also proves several theorems
about the evaluator that obviate the need for the user to reason about
the specifics of the definitions of
\texttt{meta\-/extract\-/contextual\-/fact} and
\texttt{meta\-/extract\-/global\-/fact+} and the proper construction of
their \texttt{obj} arguments.  For example, this rule shows that
\texttt{(evl\-/meta\-/extract\-/global\-/facts)} implies that a theorem looked
up from the world using \texttt{meta\-/extract\-/formula} is correct:
\begin{verbatim}
 (defthm evl-meta-extract-formula
   (implies (and (evl-meta-extract-ev-global-facts)
                 (equal (w st) (w state)))
            (evl (meta-extract-formula name st) a)))
 
\end{verbatim}
This rule shows that
\texttt{(evl\-/meta\-/extract\-/contextual\-/facts a)} implies the correctness
of \texttt{mfc-rw+} (specifically, when \texttt{nil}, meaning
\texttt{equal}, is given as the equivalence relation argument):
\begin{verbatim}
 (defthm evl-meta-extract-rw+-equal
    (implies (evl-meta-extract-contextual-facts a)
             (equal (evl (mfc-rw+
                          term alist obj nil
                          mfc state :forcep nil)
                         a)
                    (evl (sublis-var alist term) a))))
\end{verbatim}

The theorems provided by \texttt{def\-/meta\-/extract} support the following:
\begin{itemize}
\item Checking the typeset of a term with \texttt{mfc-ts}
\item Rewriting a term with \texttt{mfc-rw}
  (under \texttt{equal}, \texttt{iff}, or other equivalence relations)
\item Rewriting a term under a substitution with \texttt{mfc-rw+}
  (under \texttt{equal}, \texttt{iff}, or other equivalence relations)
\item Proving a term false via linear arithmetic with \texttt{mfc-ap}
\item Proving a term true via rewriting with \texttt{mfc-relieve-hyp}
\item Retrieving a formula from the world with \texttt{meta\-/extract\-/formula}
\item Retrieving a rewrite rule formula from a function's \texttt{lemmas} property
\item Evaluating a function call with \texttt{magic\-/ev\-/fncall}.
\end{itemize}
Comprehensive documentation is available in the ACL2 combined manual.



\section{Applications}
\label{sec:applications}
% \begin{comment}
% [Old comment] SOL will write this (perhaps including an application
%   from Centaur)
% --- Sol, is this done now?
% \end{comment}

% Since we used upper-case for subsections in intro.tex, for
% consistency let's use them in this file, too.  (But we could use
% lower-case in both; fine with me either way.)

\begin{comment}
  I was thinking of this as some examples readers could look at to
  show how meta-extract is used in practice, but now that we have the
  examples from the previous section maybe that shouldn't be the
  focus.  Instead, maybe we really want to say what the most important
  uses of it so far have been, and just show that it really is useful
  in building sophisticated tools.  So, de-emphasize the just-expand
  utility and spend a bit more time on GL?

  [Matt, replacing what I wrote in my 12/28 email] I'm probably not
  understanding your point, since to me, ``how meta-extract is used in
  practice'' and ``most important uses of it so far'' are almost the
  same.  Maybe your point was that it's better to focus on the most
  {\em important} uses than it is to focus on the practicality of a
  wide {\em range} of uses.  If so, I guess I sort of agree if I had
  to choose; but I think both are really good.  And really, the
  section already fleshes out the range of uses while mentioning some
  important ones, notably GL.  Anyhow, I'll probably be happy with
  whatever you'd like to do here.  In particular, if you care to
  provide a bit more detail about how meta-extract supports GL, that
  would be fine; but I can live with what we already have, too.  Oh,
  and it looks like you might be concerned about overlap between this
  section and the previous two sections, but I think each section
  makes its own contribution: Section~\ref{sec:meta-extract} shows how
  meta-extract works, Section~\ref{sec:user} shows your
  bad-guy-based tools, and this section shows wide applicability,
  sometimes providing important capabilities.
\end{comment}

\subsection{Just-expand Utility}

The community book ``clause-processors/just-expand.lisp'' provides a
very simple illustration of the usage of meta-extract.  It uses
meta-extract only to look up a formula for a function so that it can
expand that definition.  The core functionality is in the function
\texttt{expand-this-term}.  That function looks up the rule, or the
definition of the leading function symbol if an explicit rule is not
given, using \texttt{meta-extract-formula}.  It then essentially
applies that rule as an unconditional rewrite.  First it ensures that
its body is of the form
\begin{lstlisting}
 (equal <$\mathit{lhs}$> <$\mathit{rhs}$>).
\end{lstlisting}
If so, it tries to unify the input term with $\mathit{lhs}$, obtaining
a unifying substitution $\sigma$ if successful, and returns the
substitution of $\sigma$ into $\mathit{rhs}$.  This procedure is proven
correct under a meta-extract hypothesis, which is used to reason that
the formula obtained by looking up the rule is a theorem.

\texttt{Expand-this-term} is used to define two clause processors and
a meta rule, each of which is proven correct using meta-extract
hypotheses.  The first clause processor, \texttt{just-expand-cp},
applies \texttt{expand-this-term} to all subterms of the goal clause
that match some user-provided patterns.  The second clause processor,
\texttt{expand-marked-cp}, looks for calls of an identity function
\texttt{expand-me} and expands the term contained immediately inside
each such call, removing the \texttt{expand-me} call itself.  The meta
rule, \texttt{expand-marked-meta}, triggers on \texttt{expand-me} and
similarly expands the contained terms.

\begin{comment}
  Sol, it would be interesting to know why, for the two
  clause-processors above, you used meta-extract rather than just
  making computed hints (or override hints) that :expand suitable
  terms.  I {\em do} see why computed hints wouldn't suffice if you
  were talking about metafunctions instead of clause-processors, since
  terms to be expanded can arise during rewriting.

  [Sol] One place it's used is in a hint that inducts and then
  immediately expands the calls of the given function in the induction
  conclusion but not the induction hyps.  I'm not sure how deep we
  need to go with motivation for each of these so I'm not sure if I
  should say anything about it in the paper.

  [Matt] Hmmm, but your description doesn't make it clear to me where
  the expand-me calls come from when using expand-marked-meta to
  expand certain terms produced by induction.  I agree with your
  general uncertainty about the value of saying much about
  expand-marked-meta, but as a reader I'd like to get just a quick
  sense of the flow that results in its application.
\end{comment}

\subsection{Rewrite-bounds}

The community book ``centaur/misc/bound-rewriter.lisp'' provides a
tool for solving certain inequalities: it replaces subterms of an
inequality with known bounds if those subterms are in monotonic
positions.  For example, the term $a-b$ monotonically decreases as $b$
increases, so if we wish to prove $c<a-b$ and we know $B \geq b$, then
it suffices to prove $c<a-B$.  While this example would be easily
handled by ACL2's linear arithmetic solver, there are problems that
the bound rewriter can handle easily that overwhelm ACL2's nonlinear
solver -- e.g.,
\begin{verbatim}
(implies (and (rationalp a) (rationalp b) (rationalp c)
              (<= 0 a) (<= 0 b) (<= 1 c)
              (<= a 10) (<= b 20) (<= c 30))
         (<= (+ (* a b c) (* a b) (* b c) (* a c))
             (+ (* 10 20 30) (* 10 20) (* 20 30) (* 10 30))))
\end{verbatim}
On this formula, a \texttt{:nonlinearp t} hint causes ACL2 to hang indefinitely, while the hint
\begin{verbatim}
(rewrite-bounds ((<= a 10)
                 (<= b 20)
                 (<= c 30)))
\end{verbatim}
solves it instantaneously, by replacing upper-boundable occurrences of
\texttt{a} by 10, \texttt{b} by 20, and \texttt{c} by 30.  The same
results are obtained --- a quick proof using {\tt rewrite-bounds} but
an indefinite hang using nonlinear arithmetic --- if the arithmetic
expression on the last line of the theorem is replaced by its value,
7100.

The bound rewriter tool is implemented as a meta rule and uses
meta-extract extensively.  To determine which subterms are in
monotonic positions, it uses type-set reasoning to determine the signs
of subterms.  For example, $a \cdot b$ increases as $b$ increases if
$a$ is nonnegative and decreases as $b$ increases if $a$ is
nonpositive; if we can't (weakly) determine the sign of $a$, then we
can't replace $b$ or any subterm with a bound.  To determine whether a
proposed bound of a subterm is (contextually) true, it uses
\texttt{mfc-relieve-hyp} to show it by rewriting, and if that fails,
\texttt{mfc-ap} to show it by linear arithmetic reasoning.  The
correctness of these uses of ACL2 reasoning utilities are justified by
meta-extract-contextual-fact hypotheses.

\subsection{Others}

Several other clause processors in the community books use
meta-extract solely to be able to extract a formula from the world
(using \texttt{meta-extract-formula}) and assume it to be true.  For
example, ``clause-processors/witness-cp.lisp'' provides a framework
for reasoning about quantification (see
\href{http://www.cs.utexas.edu/users/moore/acl2/manuals/current/manual/index.html?topic=ACL2\_\_\_\_WITNESS-CP}{\underline{witness-cp}}); it uses
\texttt{meta-extract-formula} to look up a stored fact showing that a
term representing a universal quantification implies any instance
of the quantified formula.  Another book,
``clause-processors/replace-equalities.lisp'' provides a tool for
replacing known equalities in ways that aren't accessible to the
rewriter, e.g., replacing a variable with a term.  For example, the
following is not a valid rewrite rule because its left-hand side is a variable,
but it could be a good replace-equalities rule:
\begin{verbatim}
 (implies (unify-ok pattern term alist)
          (equal term
                 (subsitute pattern (unify-subst pattern term alist))))
\end{verbatim}

\begin{comment}
Above, ``subsitute'' looks like a typo.  Probably you meant
``substitute'', but probably, you really meant
``substitute-into-term''.  When I read this I thought it said
``substitute'', and I thought it was weird to use a generic sort of
list utility to create a term.  So I looked at
clause-processors/replace-equalities.lisp and found
substitute-into-term.  Maybe the reader could do that too, but I very
slightly prefer just using substitute-into-term.  You get to decide --
but of course something needs to be done about the typo,
``subsitute''.
\end{comment}

A meta rule for context-sensitive rewriting, accomplishing something
similar to Greve's ``Nary'' framework \cite{greve06}, is defined in
``centaur/misc/context-rw.lisp'' (see
\href{http://www.cs.utexas.edu/users/moore/acl2/manuals/current/manual/index.html?topic=ACL2\_\_\_\_CONTEXTUAL-REWRITING}{\underline{contextual-rewriting}}).
It uses meta-extract to allow it to
trust \texttt{mfc-relieve-hyp}, \texttt{mfc-rw+}, and
\href{http://www.cs.utexas.edu/users/moore/acl2/manuals/current/manual/index.html?topic=ACL2\_\_\_\_MAGIC-EV-FNCALL}{\underline{\tt magic-ev-fncall}}.

The
\href{http://www.cs.utexas.edu/users/moore/acl2/manuals/current/manual/index.html?topic=ACL2\_\_\_\_GL}{\underline{GL}}
framework for bit-level symbolic execution~\cite{gl-diss,
  bit-blasting-GL} uses meta-extract to look up function definitions
and rewrite rules from the world and to evaluate ground terms using
\texttt{magic-ev-fncall}.


% No related work section; see top of related.tex
% \section{Related Work}
% \label{sec:related}
% NOTE: Matt has expanded the first paragraph of the introduction to
mention some related work, and suggests that we not have a related
work section.  The wording in the intro now makes it clear (hopefully)
that this paper is about an extension, so presumably related work
would be found in preceding papers.

{\it \color{red} MATT will write this (but will welcome any ideas from
  Sol):

\begin{enumerate}

\item Boyer-Moore paper~\cite{meta}
\item ACL2 paper on meta~\cite{meta-05}
\item Milawa?~\cite{davis09}

\end{enumerate}

Probably we should give a nod to other ITP systems that do meta, but
I'm not sure and I don't know what they are.  Maybe I'll look in
\cite{meta-05}.  I thought of pointing to Harrison's book, but the
table of contents at http://www.gbv.de/dms/ilmenau/toc/585835187.PDF
seem not to have anything about meta.

% @book{Harrison:2009:HPL:1540610,
%  author = {Harrison, John},
%  title = {Handbook of Practical Logic and Automated Reasoning},
%  year = {2009},
%  isbn = {0521899575, 9780521899574},
%  edition = {1st},
%  publisher = {Cambridge University Press},
%  address = {New York, NY, USA},
% }

}


\section{Conclusion}
\label{sec:conclusion}
\begin{mycomment}
[Old comment] One or both of us will write this.  Maybe we'll wait
  till a draft of the rest is written, so that we can hit the high
  points here.  Any ideas about future work?  Any lessons learned?
  (It would be cool if some application would likely have been much
  more difficult to carry out without this.)
\end{mycomment}

\begin{mycomment}
Is there a grep command we can run in books showing/ that we use
    meta-extract?  I tried

\begin{verbatim}
time fgrep --include='*.lisp' -ri meta-extract . | fgrep -v system/doc/acl2-doc.lisp
\end{verbatim}

\noindent but it includes things like {\tt
boundrw-ev-meta-extract-contextual-facts} and {\tt
ctx-ev-meta-extract-contextual-facts}, which I don't understand.

Below is an initial stab at the conclusion by Matt, which undoubtedly
could be improved.  Maybe you can say a bit more, at a high level,
    about meta-extract at Centaur (even if only referring back to the
    applications section).
\end{mycomment}

This paper explains meta-extract hypotheses and shows how they can be
put to good use, either directly or by way of the {\tt
def-meta-extract} utility described in Section~\ref{sec:user}.

The history of ACL2 is marked by increasingly sophisticated
applications of it.  Many advanced features of ACL2 are in regular use
at Centaur Technology, including meta-extract hypotheses.  We hope
that this paper contributes to wider successful use of that feature.


\section*{Acknowledgments}

% Matt had some chats with J about this stuff back in 2012.
We thank J Moore for helpful discussions.
We also thank the referees for useful feedback.
This material is based upon work supported in part by DARPA under
Contract No. FA8750-15-C-0007 (subcontract 15-C-0007-UT-Austin)
and by ForrestHunt, Inc.

% \nocite{*}
\bibliographystyle{eptcs}
\bibliography{paper}
\end{document}


