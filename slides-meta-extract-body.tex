% Another file will have the \documentclass and then \input this one.
% \documentclass[xcolor=x11names,compress]{beamer}

%% General document %%%%%%%%%%%%%%%%%%%%%%%%%%%%%%%%%%
\usepackage[english]{babel}
\usepackage[utf8]{inputenc}
\usepackage{times}
\usepackage[T1]{fontenc}
\usepackage{lipsum}
\usepackage{ragged2e}
\usepackage{alltt}
\usepackage{rotating}
\usepackage{lscape}
\usepackage{setspace}
\usepackage{array}
\usepackage{verbatim}
\usepackage{fancyvrb}
\usepackage{color}
\usepackage{xspace}
\newcommand\codeHighlightr[1]{\textcolor[rgb]{1,0,0}{\textbf{#1}}}
\newcommand\codeHighlightg[1]{\textcolor[rgb]{0,1,0}{\textbf{#1}}}
\newcommand\codeHighlightb[1]{\textcolor[rgb]{0,0,1}{\textbf{#1}}}
\newcommand{\SmallSkip}{\vspace{0.5cm}\noindent}
\newcommand{\demo}{{\color{red} [DEMO]}}
\definecolor{darkorange}{RGB}{210,105,30}
\usepackage[english]{babel}
%%%%%%%%%%%%%%%%%%%%%%%%%%%%%%%%%%%%%%%%%%%%%%%%%%%%%%


%% Beamer Layout %%%%%%%%%%%%%%%%%%%%%%%%%%%%%%%%%%
\useoutertheme[subsection=false,shadow]{miniframes}
\useinnertheme{default}
\usefonttheme{serif}
\usepackage{palatino}

\setbeamerfont{title like}{shape=\scshape}
\setbeamerfont{frametitle}{shape=\scshape}

%\setbeamercolor*{lower separation line head}{bg=DeepSkyBlue4}
\setbeamercolor*{lower separation line head}{bg=darkorange}
\setbeamercolor*{normal text}{fg=black,bg=white}
\setbeamercolor*{alerted text}{fg=red}
\setbeamercolor*{example text}{fg=black}
\setbeamercolor*{structure}{fg=black}

\setbeamercolor*{palette tertiary}{fg=black,bg=black!10}
\setbeamercolor*{palette quaternary}{fg=black,bg=black!10}

\renewcommand{\(}{\begin{columns}}
\renewcommand{\)}{\end{columns}}
\newcommand{\<}[1]{\begin{column}{#1}}
\renewcommand{\>}{\end{column}}
\setbeamertemplate{footline}{\hfill\insertframenumber/\inserttotalframenumber}
\setbeamertemplate{navigation symbols}{}

%\setbeamersize{text margin right=20pt}

% Modified from Jared (xdoc paper):
\hypersetup{%
   colorlinks=true,% hyperlinks will be black
   linkcolor=blue,% added by Matt K.
   urlcolor=blue,% added by Matt K.
   urlbordercolor=blue,% hyperlink borders will be red
   pdfborderstyle={/S/U/W 1.5}% border style will be underline of width 1pt
}

\definecolor{darkcyan}{cmyk}{1.0,0.2,0.2,0.2}

\newcommand\mydotdot{{\color{orange} ...}}

\AtBeginSection[]
{
   \begin{frame}
       \frametitle{\hspace{4mm}Outline}
       \tableofcontents[currentsection, currentsubsection]
   \end{frame}
}

%%%%%%%%%%%%%%%%%%%%%%%%%%%%%%%%%%%%%%%%%%%%%%%%%%

\begin{document}

%%%%%%%%%%%%%%%%%%%%%%%%%%%%%%%%%%%%%%%%%%%%%%%%%%%%%%
\begin{frame}
\title{Meta-extract: Using Existing Facts in Meta-reasoning}
%\subtitle{SUBTITLE}
\author{
        Matt Kaufmann (UT Austin) \\
        Sol Swords (Centaur Technology)
}
\date{
        \vspace{1cm}
        ACL2 Workshop 2017
}
\titlepage
\end{frame}
%%%%%%%%%%%%%%%%%%%%%%%%%%%%%%%%%%%%%%%%%%%%%%%%%%%%%%
\begin{frame}{\hspace{4mm}Outline}
\tableofcontents
\end{frame}
%%%%%%%%%%%%%%%%%%%%%%%%%%%%%%%%%%%%%%%%%%%%%%%%%%%%%%
%%%%%%%%%%%%%%%%%%%%%%%%%%%%%%%%%%%%%%%%%%%%%%%%%%%%%%
\section{\scshape Introduction}
\begin{frame}[fragile]
\frametitle{Introduction}

\begin{itemize}

\item ACL2 supports rule classes {\tt :meta} and {\tt
    :clause-processor} that allow user-supplied extensions to the
  prover.
  
  \begin{itemize}
  \item In this talk we'll focus on {\tt :meta} rules and corresponding
    {\em metafunctions}.
  \end{itemize}
  
\item The {\em meta-extract} utilities (introduced in late 2012)
  permit correctness of metafunctions to depend on facts extracted
  from the {\em world} or the {\em context}.

  \begin{itemize}
  \item Metafunctions can access both; clause processors have only a world, not a context.
  \item The facts will be \textit{extracted} at runtime, but \textit{assumed correct} when proving the metafunction.
  \item Therefore, routines can use facts that were introduced \textit{after} they were proved correct.
  \end{itemize}
\end{itemize}
\end{frame}

\begin{frame}[fragile]
\frametitle{This talk}
  \begin{itemize}

  \item reviews meta reasoning;

  \item gives two simple examples to illustrate meta-extract hypotheses;

  \item discusses a nice short-cut;

  \item summarizes some applications; and

  \item points to the paper and supporting materials for more information.

  \end{itemize}

\end{frame}
%%%%%%%%%%%%%%%%%%%%%%%%%%%%%%%%%%%%%%%%%%%%%%%%%%%%%%
%%%%%%%%%%%%%%%%%%%%%%%%%%%%%%%%%%%%%%%%%%%%%%%%%%%%%%
\section[\scshape Review]{\scshape Review of {\tt :Meta} Rules}
\begin{frame}[fragile]{Review of {\tt :Meta} Rules}

The canonical example of a {\tt :meta} rule is
\verb|cancel_plus-equal|, from
community book {\tt meta/meta-plus-equal}.

It cancels like terms from the equality of two sums.

{\footnotesize
\begin{verbatim}
ACL2 !>:trans (equal (+ x y x z) (+ x z z z))

(EQUAL (BINARY-+ X (BINARY-+ Y (BINARY-+ X Z)))
       (BINARY-+ X (BINARY-+ Z (BINARY-+ Z Z))))

=> *

ACL2 !>(cancel_plus-equal
        '(EQUAL (BINARY-+ X (BINARY-+ Y (BINARY-+ X Z)))
                (BINARY-+ X (BINARY-+ Z (BINARY-+ Z Z)))))
(EQUAL (BINARY-+ Y X) (BINARY-+ Z Z))
ACL2 !>
\end{verbatim}
}

\end{frame}
%%%%%%%%%%%%%%%%%%%%%%%%%%%%%%%%%%%%%%%%%%%%%%%%%%%%%%
\begin{frame}[fragile]{Review of {\tt :Meta} Rules (2)}

Key events:
\begin{itemize}
\item Define an evaluator:
\begin{verbatim}
(defevaluator ev-plus-equal ...)
\end{verbatim}
  
\item Define the metafunction:
\begin{verbatim}
(defun cancel_plus-equal (x)
  (declare (xargs :guard (pseudo-termp x)))
  ...)
\end{verbatim}

\item Prove the metafunction correct w.r.t. the evaluator:
\begin{verbatim}
(defthm cancel_plus-equal-correct
  (equal
   (ev-plus-equal x a)
   (ev-plus-equal (cancel_plus-equal x) a))
  :rule-classes ((:meta :trigger-fns (equal))))
\end{verbatim}
\end{itemize}
Let's see this rule used in a proof.

\end{frame}
%%%%%%%%%%%%%%%%%%%%%%%%%%%%%%%%%%%%%%%%%%%%%%%%%%%%%%
\begin{frame}[fragile]{Review of {\tt :Meta} Rules (2)}

% {\scriptsize
{\footnotesize
\begin{verbatim}
ACL2 !>(include-book "meta/meta-plus-equal" :dir :system)
....
ACL2 !>(trace$ cancel_plus-equal)
 ((CANCEL_PLUS-EQUAL))
ACL2 !>(thm (implies (and (acl2-numberp z)
                          (equal (+ x y x z) (+ x z z z)))
                     (equal z (/ (+ x y) 2)))
            :hints (("Goal" :in-theory (disable (tau-system)))))
Goal'
1> (CANCEL_PLUS-EQUAL
    (EQUAL (BINARY-+ X (BINARY-+ X (BINARY-+ Y Z)))
           (BINARY-+ X (BINARY-+ Z (BINARY-+ Z Z)))))
<1 (CANCEL_PLUS-EQUAL (EQUAL (BINARY-+ X Y) (BINARY-+ Z Z)))
....
Proof succeeded.
ACL2 !>
\end{verbatim}
}

\end{frame}
%%%%%%%%%%%%%%%%%%%%%%%%%%%%%%%%%%%%%%%%%%%%%%%%%%%%%%
%%%%%%%%%%%%%%%%%%%%%%%%%%%%%%%%%%%%%%%%%%%%%%%%%%%%%%
\section[\scshape Example 1]{\scshape Example 1: Using Contexts}
\begin{frame}[fragile]{Example 1: Using Contexts}

Consider this metafunction:
\begin{Verbatim}[frame=none, framesep=5pt]
(defun nth-symbolp-metafn (term mfc state)
  (declare (xargs :stobjs state))
  (case-match term
    (('nth n x)
     (if (equal (mfc-ts n mfc state :forcep nil)
                *ts-symbol*)
         (list 'car x)
       term))
    (& term)))
\end{Verbatim}
Approximately: ``If term is \verb|(nth n x)| and \verb|n| is
known to be a symbol in the current context, replace it with
\verb|(car x)|.''

\end{frame}
%%%%%%%%%%%%%%%%%%%%%%%%%%%%%%%%%%%%%%%%%%%%%%%%%%%%%%

\begin{frame}[fragile]{Example 1: Using Contexts}
\begin{itemize}
\item How can we prove this correct?
\item Before meta-extract we'd need to somehow verify that \texttt{mfc-ts} was ``telling the truth''
  \begin{itemize}
  \item E.g., have a hypothesis metafunction that produces the corresponding assumption.
  \end{itemize}
\item Meta-extract lets you assume this while proving your metafunction correct.
\end{itemize}
\end{frame}
%%%%%%%%%%%%%%%%%%%%%%%%%%%%%%%%%%%%%%%%%%%%%%%%%%%%%%

\begin{frame}[fragile]{Example 1: Using Contexts}
Correctness theorem for \texttt{nth-symbolp-metafn}:

\begin{Verbatim}[formatcom=\small]
; workshops/2017/kaufmann-swords/support/intro.lisp
(defthm nth-symbolp-meta
  (implies
   ;; Meta-extract hypothesis:
   (nthmeta-ev (meta-extract-contextual-fact
                 `(:typeset ,(cadr term))
                 mfc 
                 state)
               a)
   ;; Standard meta rule conclusion:
   (equal (nthmeta-ev term a)
          (nthmeta-ev (nth-symbolp-metafn
                       term mfc state)
                      a)))
  :rule-classes ((:meta :trigger-fns (nth))))
\end{Verbatim}

\end{frame}
%%%%%%%%%%%%%%%%%%%%%%%%%%%%%%%%%%%%%%%%%%%%%%%%%%%%%%
\begin{frame}[fragile]{Example 1: Meta-extract Hypothesis}
  \texttt{Meta-extract-contextual-fact}:
\begin{itemize}
\item Returns various terms expressing facts known under a given context.
\item Only produce terms that are known true.
\item Meta rule theorems are allowed to assume the terms it produces
  evaluate to true.
\end{itemize}
Part of the definition:
\begin{Verbatim}[formatcom=\small]
(case-match obj
  ((':typeset term . &) ; mfc-ts produces correct results
   `(typespec-check
     ',(mfc-ts term mfc state :forcep nil :ttreep nil)
     ,term))
\end{Verbatim}
\end{frame}
%%%%%%%%%%%%%%%%%%%%%%%%%%%%%%%%%%%%%%%%%%%%%%%%%%%%%%
\begin{frame}[fragile]{Meta-Extract-Contextual-Fact}
Supports:
\begin{itemize}
\item Typeset reasoning (\texttt{mfc-ts})
\item Rewriting (\texttt{mfc-rw},\ \ \texttt{mfc-rw+},\ \ \texttt{mfc-relieve-hyp})
\item Linear arithmetic (\texttt{mfc-ap})
\end{itemize}
\end{frame}

%%%%%%%%%%%%%%%%%%%%%%%%%%%%%%%%%%%%%%%%%%%%%%%%%%%%%%
%%%%%%%%%%%%%%%%%%%%%%%%%%%%%%%%%%%%%%%%%%%%%%%%%%%%%%
\section[\scshape Example 2]{\scshape Example 2: Using Global Facts}
\begin{frame}[fragile]{Example 2: Using Global Facts}

Goal: Rewrite stobj \texttt{(accessor (updater val foo\$))} terms without either:
\begin{itemize}
\item proving $n^2$ individual rules per stobj
\item enabling accessors/updaters to just expand to \texttt{nth}
\end{itemize}
An approach: \texttt{nth-update-nth-ev-meta-fn} checks that
\texttt{accessor} is defined as a call of \texttt{nth} and
\texttt{updater} is defined as a call of \texttt{update-nth} and
rewrites accordingly.
\end{frame}
%%%%%%%%%%%%%%%%%%%%%%%%%%%%%%%%%%%%%%%%%%%%%%%%%%%%%%
\begin{frame}[fragile]{Example 2: Using Global Facts}
  Key: Function definitions can be retrieved from the world using
  \texttt{meta-extract-formula} and \textit{assumed correct} using
  a \texttt{meta-extract-global-fact} hyp.
\end{frame}
%%%%%%%%%%%%%%%%%%%%%%%%%%%%%%%%%%%%%%%%%%%%%%%%%%%%%% 
\begin{frame}[fragile]{Example 2: Using Global Facts}

\begin{verbatim}
; demos/nth-update-nth-meta-extract.lisp
(defthm nth-update-nth-meta-rule-st
  (implies
   (and (nth-update-nth-ev ; (f (update-g val st))
         (meta-extract-global-fact
          (list :formula (car term)) state)
         (meta-extract-alist term a state))
        ...)
   (equal (nth-update-nth-ev term a)
          (nth-update-nth-ev
           (nth-update-nth-meta-fn term mfc state)
           a)))
  :hints ...
  :rule-classes ((:meta :trigger-fns ...)))
\end{verbatim}

\end{frame}
%%%%%%%%%%%%%%%%%%%%%%%%%%%%%%%%%%%%%%%%%%%%%%%%%%%%%% 
\begin{frame}[fragile]{Meta-Extract-Global-Fact}
Supports:
\begin{itemize}
\item Theorem bodies, function definitions, and constraints (\texttt{meta-extract-formula})
\item Rewrite rules from functions' \texttt{lemmas} properties
\item Evaluation of ground function calls (\texttt{magic-ev-fncall}).
\end{itemize}
\end{frame}
%%%%%%%%%%%%%%%%%%%%%%%%%%%%%%%%%%%%%%%%%%%%%%%%%%%%%% 
%%%%%%%%%%%%%%%%%%%%%%%%%%%%%%%%%%%%%%%%%%%%%%%%%%%%%%
\section[\scshape Shortcut]{\scshape A Nice Short-cut}
\begin{frame}[fragile] {A Nice Short-cut}
Community book ``clause-processors/meta-extract-user'' provides a utility
\texttt{def-meta-extract}, 
The utility {\tt def-meta-extract}, which defines two
hypotheses that imply all contextual and global facts, respectively.

\begin{itemize}

\item {\tt (evl-meta-extract-contextual-facts a)}

\item {\tt (evl-meta-extract-ev-global-facts)}

\end{itemize}

This is provided by community book {\tt
  clause-processors/meta-extract-user.lisp}, thus minimizing the ACL2
trusted code base.

Here is an example showing the power of (one of) these hypotheses.

\begin{verbatim}
(defthm evl-meta-extract-formula
   (implies (and (evl-meta-extract-ev-global-facts)
                 (equal (w st) (w state)))
            (evl (meta-extract-formula name st) a)))
\end{verbatim}

\end{frame}
%%%%%%%%%%%%%%%%%%%%%%%%%%%%%%%%%%%%%%%%%%%%%%%%%%%%%%
%%%%%%%%%%%%%%%%%%%%%%%%%%%%%%%%%%%%%%%%%%%%%%%%%%%%%%
\section[\scshape Applications]{\scshape Some Applications}
\begin{frame}[fragile] {Some Applications}

\begin{itemize}

\item The GL symbolic interpreter uses meta-extract hypotheses to
  call functions without additional proof obligations.

\item The community book {\tt centaur/misc/bound-rewriter.lisp}
  provides a tool for solving certain inequalities:

\item A meta rule for context-sensitive rewriting (like Greve's
  ``nary'' framework is defined in {\tt centaur/misc/context-rw.lisp}.

\item Others....

\end{itemize}

\end{frame}
%%%%%%%%%%%%%%%%%%%%%%%%%%%%%%%%%%%%%%%%%%%%%%%%%%%%%%
%%%%%%%%%%%%%%%%%%%%%%%%%%%%%%%%%%%%%%%%%%%%%%%%%%%%%%
\section{\scshape Conclusion}
\begin{frame}[fragile] {Conclusion}

Some concluding thoughts....

\begin{itemize}

\item This talk is just an introduction; meta reasoning is a bit
  complex to absorb in real time!

\item The paper develops the ideas from this talk more thoroughly,
  with more illustrative examples.  See also: \\
  {\tt books/workshops/2017/kaufmann-swords/support/README}.

\item If you use GL then you are already taking advantage of
  meta-extract.

\end{itemize}

\end{frame}
%%%%%%%%%%%%%%%%%%%%%%%%%%%%%%%%%%%%%%%%%%%%%%%%%%%%%%
%%%%%%%%%%%%%%%%%%%%%%%%%%%%%%%%%%%%%%%%%%%%%%%%%%%%%%
\end{document}
