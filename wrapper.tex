The community book ``clause-processors/meta-extract-user.lisp'' is
designed to allow more convenient use of the meta-extract facility.
The main contribution of this book is in the event-generating macro
\texttt{def\-/meta\-/extract}. For a given pseudo-evaluator \texttt{evl},
\texttt{def\-/meta\-/extract} produces macros
\texttt{evl\-/meta\-/extract\-/contextual\-/facts} and
\texttt{evl\-/meta\-/extract\-/global\-/facts} that expand to meta-extract
hypotheses where the \texttt{obj} argument is a call of a ``bad-guy''
function.  This essentially universally quantifies the \texttt{obj}
argument:
\begin{verbatim}
  (evl (meta-extract-contextual-fact
         (evl-meta-extract-bad-guy a mfc state)
         mfc state)
       a)
\end{verbatim}
implies for any \texttt{obj},
\begin{verbatim}
  (evl (meta-extract-contextual-fact obj state) a).
\end{verbatim}

The \texttt{def-meta-extract} utility also proves several theorems
about the evaluator that obviate the need for the user to reason about
the specifics of the definitions of
\texttt{meta\-/extract\-/contextual\-/fact} and
\texttt{meta\-/extract\-/global\-/fact+} and the proper construction of
their \texttt{obj} arguments.  For example, this rule shows that
\texttt{(evl\-/meta\-/extract\-/global\-/facts)} implies that a theorem looked
up from the world using \texttt{meta\-/extract\-/formula} is correct:
\begin{verbatim}
 (defthm evl-meta-extract-formula
   (implies (and (evl-meta-extract-ev-global-facts)
                 (equal (w st) (w state)))
            (evl (meta-extract-formula name st) a)))

\end{verbatim}
This rule shows that
\texttt{(evl\-/meta\-/extract\-/contextual\-/facts a)} implies the correctness
of \texttt{mfc-rw+} (specifically, when \texttt{nil}, meaning
\texttt{equal}, is given as the equivalence relation argument):
\begin{verbatim}
 (defthm evl-meta-extract-rw+-equal
    (implies (evl-meta-extract-contextual-facts a)
             (equal (evl (mfc-rw+
                          term alist obj nil
                          mfc state :forcep nil)
                         a)
                    (evl (sublis-var alist term) a))))
\end{verbatim}

The theorems provided by \texttt{def\-/meta\-/extract} support the following:
\begin{itemize}
\item Checking the typeset of a term with \texttt{mfc-ts}
\item Rewriting a term with \texttt{mfc-rw}
  (under \texttt{equal}, \texttt{iff}, or other equivalence relations)
\item Rewriting a term under a substitution with \texttt{mfc-rw+}
  (under \texttt{equal}, \texttt{iff}, or other equivalence relations)
\item Proving a term false via linear arithmetic with \texttt{mfc-ap}
\item Proving a term true via rewriting with \texttt{mfc-relieve-hyp}
\item Retrieving a formula from the world with \texttt{meta\-/extract\-/formula}
\item Retrieving a rewrite rule formula from a function's \texttt{lemmas} property
\item Evaluating a function call with \texttt{magic\-/ev\-/fncall}.
\end{itemize}
Comprehensive documentation is available in the ACL2 combined
manual~\cite{acl2:doc}.
