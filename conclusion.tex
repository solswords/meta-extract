\begin{comment}
[Old comment] One or both of us will write this.  Maybe we'll wait
  till a draft of the rest is written, so that we can hit the high
  points here.  Any ideas about future work?  Any lessons learned?
  (It would be cool if some application would likely have been much
  more difficult to carry out without this.)
\end{comment}

\begin{comment}
Is there a grep command we can run in books showing/ that we use
    meta-extract?  I tried

\begin{verbatim}
time fgrep --include='*.lisp' -ri meta-extract . | fgrep -v system/doc/acl2-doc.lisp
\end{verbatim}

\noindent but it includes things like {\tt
boundrw-ev-meta-extract-contextual-facts} and {\tt
ctx-ev-meta-extract-contextual-facts}, which I don't understand.

[Sol] These are macros for the universally-quantified hyps produced by
{\tt def-meta-extract}.  I think that's a good way to look for the
books that define metafunctions/clause-processors using meta-extract;
I did something like that to find examples for Section
\ref{sec:applications}.

Below is an initial stab at the conclusion by Matt, which undoubtedly
could be improved.  Maybe you can say a bit more, at a high level,
    about meta-extract at Centaur (even if only referring back to the
    applications section).
\end{comment}

This paper explains meta-extract hypotheses and shows how they can be
put to good use, either directly or by way of the {\tt
  def-meta-extract} utility described in Section~\ref{sec:user}.  This
facility has been successfully used to create specialized proof tools
that are admitted by ACL2 as fully verified metafunctions or
clause-processors, including the GL bit-blasting framework, which is
in daily use at Centaur Technology.  We hope that this paper
contributes to wider successful use of the meta-extract feature.
